\pdfbookmark[0]{Abstract}{Abstract}
\chapter*{Abstract}
\label{sec:abstract}
\vspace*{-10mm}
Besonders im Amateurbereich spielt die Vorbereitungsphase zu den Wettkämpfen eine wichtige Rolle. Die maximale Leistung auf dem Rad wird vorallem von der physischen Leistung beeinflusst. Unterschiedliche Trainingsziele und Leistungsstände nehmen Einfluss auf die Trainingsinhalte eines individuellen Sportlers. Mit personalisierten Trainingsplänen soll ein optimaler Trainingsverlauf für eine breite Personengruppe ermöglicht werden.
Um die Leistungssteigerung zu optimieren, wird diese Modellierung nach Erkenntnissen der allgemeinen und radsportspezifischen Trainingswissenschaft Trainingseinheiten aufeinander abstimmen.


\begin{itemize}
    \item Eventuell Verbindung zu Fahrradboom durch Corona
    \item Oder generell steigender Anzahl an Fahrradfahren. Aber Fahrrad fahren als Transportmittel $\neq$ Sport
\end{itemize}
\vspace*{20mm}

{\usekomafont{chapter}Abstract (english)?}\label{sec:abstract-diff} \\
\vspace*{-10mm}
\clearpage

{\usekomafont{chapter}Notizen}
\begin{itemize}
    \item Bei Vorbereitung auf einen einzelnen Wettkampf, Streckenverlauf einbeziehen?
    \item Wie die optimale Leistungsbelastung bestimmen ohne Leistungsmesser? Ist das überhaupt nötig, wenn keine Trainingskontrolle stattfinden soll?
    \item Einfluss von Ausrüstung - Fahrradschuhe
    \item Test um Leistungsstand zu ermitteln?
    \item Intensität in einer Trainingseinheit variieren? Angaben zur Intensitätsstufe/min
    \item Zielgruppe bestimmen (Amateur, Profis haben keine richtige Vorbereitungszeit?, ...). Reflektiert das der Trainingsumfang schon?
    \item vorrausgehende Tests um den aktuellen Trainingsstand zu ermitteln?
    \item Übertraining verhindern durch Regenerationssteuerung. Es werden keine Messungen/Anpassung im Trainingsverlauf vorgenommen, sodass Übertrainingszustände durch Einplanen von ausreichenden Regenerationsphasen verhindert werden sollen.
    \item Leistungsnachlass durch zu lange Pausen/Regenerationsphasen verhindern (weniger als 5 Tage und 10 Tage)
\end{itemize}
\chapter{Trainingswissenschaftliche Grundlagen (15 Seiten)}
\label{sec:grundlagen}
Vor der Erstellung eines Modells gilt es die Merkmale eines optimalen Trainingsverlaufs zu spezifizieren. Dabei gibt es allgemeingültige sowie sportartspezifische Richtlinien für die Gestaltung eines Trainingsjahres sowie darin enthaltener konkreter Trainingseinheiten.
Die hierarchische Struktur einer Trainingsplans kann auf jeder Ebene periodisch und zyklisch gestaltet werden.

\section{Trainingsprinzipien}
\subsection{progressive Trainingsbelastung}
    \cite[60-61]{Radsporttraining}
    steigende Belastung im Jahresverlauf aber auch in einzelnen Perioden
\subsection{Regeneration}
    \begin{itemize}
        \item Länge abhängig von Intensität+Dauere vorangegangener Leistung
        \item nach Wettkämpfen eine Phase einplanen
    \end{itemize}

\subsection{Superkompensationsmethode}
    \cite[44-46]{Radsporttraining}
    Erholungsphase optimal platzieren um Phase der erhöhten Leistungsbereitschaft zu nutzen
    \begin{itemize}
        \item Nur ein Training pro Tag
        \item Belastung im Block (aufeinanderfolgende Tage)
        \item Erholungstag (= maximal Kompensationstraining) nach Belastungsblock
        \item Krafttrainings nach dem Erholungstag
    \end{itemize}
    
\subsection{Periodisierung}
Die Periodisierung charakterisiert die phasenförmige Veränderung von Teilzielen, Inhalten, Methoden und Organisationsformen im Jahrestrainingsaufbau.
Im Falle eines wettkampforientierten Trainingsjahres sollte zum Wettkampf die höchste Leistung erreicht werden. \cite[279]{Trainingswissenschaft}
Ob einfach, doppelt oder mehrfach periodisiert wird, steht im Zusammenhang mit der Dauer des Plans sowie der Anzahl der bevorstehenden Wettkampfsphasen.

\subsection{Zyklisierung}
Die Zyklisierung des Plans findet in allen Ebenen der Trainingsplans statt.

\TODO{eventuell nach Perioden des Plans strukturieren, da die Ziele konkreter enthalten sind?}
\section{Trainingseinheit}
\label{grundlagen:einheiten}
    \subsection{Trainingsmethoden}
        Mehr Infos in \cite[40]{Radsporttraining}
        \subsubsection{Dauerleistungsmethode}
        \subsubsection{Intervallmethode}
        \subsubsection{Wiederholungsmethode}
    \subsection{Belastungsbereiche}
    Quelle: \cite[31-39]{Radsporttraining}
    \begin{table}[h]
        \centering
        \begin{tabular}[h]{c|l|l}
            ... & Trainingsbereich & Beschreibung \\
            KB & Kompensationsbereich & niedrigste Trainingsintensität \\
            GA & Grundlagenausdauer & leichte Intensität \\
            EB & Entwicklungsbereich & \\
            SB & Spitzenbereich & \\
            K1-K2 & Maximal- und Schnellkraftbereich & \\
            K3-K4-K5 & Kraftausdauerbereich & \\
            & Wettkampfsbereich & \\
        \end{tabular}
        \caption{Bereiche der Leistungsstruktur}
        \label{tab:trainingsbereiche}
    \end{table}
    
    Andere Benennung bei \cite[27]{Ausdauertrainer} aber auch 7 Stufen
    
    \subsubsection{Kompensationsbereich (KB)}
    In \cite[46]{Ausdauertrainer} auch REKOM genannt
    
    In \cite[]{Radsporttraining}
    \begin{itemize}
        \item zur Kompensation und aktiven Regeneration vor und nach intensiven Belastungen
        \item niedrigste Trainingsintensität also HF < 60\% der max
        \item unterer Bereich des Fettstoffwechsels unter aeroben Verhältnissen
        \item \textbf{Dauer einer Fahrt < 1 bis 2 Stunden}
        \item auch als Lauftraining möglich, dann Dauer < 45 min (nicht während Rennsaison)
    \end{itemize}
    \subsubsection{Grundlagenausdauer (GA)}
    \subsubsection{Entwicklungsbereich (EB)}
    \subsubsection{Spitzenbereich (SB)}
    \subsubsection{Maximal und Schnellkraftsbereich (K1-K2)}
    \subsubsection{Kraftausdauer (K3-K4-K5)}
    \subsubsection{Wettkampfsbereich}

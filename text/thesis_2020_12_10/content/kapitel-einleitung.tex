\chapter{Einleitung (2 Seiten)}
\label{sec:einleitung}
Motivation der Arbeit
\begin{itemize}
    \item Zeit sparen Trainingspläne selbst zu schreiben
    \item Struktur und Beständigkeit ins Training bekommen 
    \item Gibt es psychologische Vorteile eines Trainingsplans? bewusste Zielsetzung/Selbstmotivation? -> Quellen nötig
\end{itemize}


\section{Problemstellung}
\label{sec:einleitung:problem}
Für den Radsport gibt es im Freizeit- und Amateurbereich Potential der Leistungssteigerung durch ein strukturelles Training. Auch wenn es vorgefertigte Standardpläne gibt, können diese nicht die individuellen Anforderungen erfüllen. Besonders zeitliche Einschränkungen sowohl im Umfang als auch in der Verfügbarkeit finden größtenteils keine Berücksichtigung.
Trotz der vielschichtigen Leistungsfaktoren behandelt diese Arbeit ausschließlich die physische Leistungssteigerung. Ziel ist es durch die Modellierung eines optimalen Trainingsplans die Erstellung solcher automatisiert und ohne Betreuer zu ermöglichen. 
Andere Einflussfaktoren wie psychische Belastung, Taktische Entscheidungen,  Ausrüstung oder Fahrbeschaffenheit \cite[13-15]{Radsporttraining} sind nicht Rahmen dieser Arbeit.

\section{Zielgruppe}
Profis nicht abgedeckt. keine rictige Vorbereitungsphase.

Amateure mit mehreren Wettkampfsphasen einbeziehen?

Hobbysportler mit einem einzelnen Wettkampf 

Hobbysportler ohne Wettkampf

\section{Überblick}
\label{sec:intro:ueberblick}
\textbf{\fullref{sec:verwandt}} \\[0.2em]

\textbf{\fullref{sec:grundlagen}} \\[0.2em]
Kapitel \ref{sec:grundlagen} befasst sich mit den Grundlagen der Sportwissenschaften sowohl im allgemeinen Bereich als auch den radsportspezifischen Anforderungen. Trainingseinheiten werden im Kontext der Periodisierung und Zyklisierung unterschiedliche Trainingsziele erfüllen und so verschiedene Belastungsbereiche abdecken.

\textbf{\fullref{sec:modellierung}}\\[0.2em]
Aus den Grundlagen der Trainingswissenschaft wird der optimale Trainingsplan modelliert. Dabei sollen individuelle Parameter des Nutzers berücksichtigt werden. Trainingsziele, Leistungsgruppen sowie wöchentlicher Umfang und verfügbare Zeiten fließen hier ein. 

\textbf{\fullref{sec:design}} \\[0.2em]
Die Modellierung wird in eine Software umgesetzt, die in Kapitel \ref{sec:design} konzipiert wird. Im UML Diagram finden sich die verwendeten Entwurfsmuster wieder. Vorrangig geht es um die Einbindung des Modells mit ILP in die Backend Anwendung. Zusätzlich entsteht die Oberfläche um die Daten zu visualisieren, die die API anbietet.

\textbf{\fullref{sec:implementierung}} \\[0.2em]
Die konkrete Umsetzung des Softwareprojekts wird als Webanwendung zur Verfügung gestellt. Die Umsetzung des Backends in Code je nach Solver.

\textbf{\fullref{sec:zusammenfassung}} \\[0.2em]
Abschließend wird die Übertragbarkeit auf andere Sportarten aber auch die Individualität der generierbaren Trainingspläne diskutiert.

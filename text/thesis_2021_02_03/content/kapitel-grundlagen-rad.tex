\chapter{Trainingswissenschaftliche Grundlagen}
\label{sec:grundlagen:rad}
Vor der Erstellung eines Modells gilt es die Merkmale eines optimalen Trainingsverlaufs zu spezifizieren. Dabei gibt es allgemeingültige sowie sportartspezifische Prinzipien der Gestaltung einer Trainingseinheit sowie darin enthaltener Trainingsabschnitte.

\section{Trainingsprinzipien}
\subsection{Periodisierung}
    Zweck eines Trainingsplans ist es in einem Zeitrahmen Einheiten einzuplanen, um das Trainingsziel zu erreichen. Daraus hergeleitete Teilziele dienen einer detaillierteren Struktur. Die Granularität kann sogar bis zu einer individuellen Ausrichtung der Woche reichen. Jedes Teilziel wird in einer Periode behandelt, die Einfluss auf Inhalt der Trainingseinheiten haben.\cite{periodization} \newline
    Im Falle eines wettkampfsorientierten Trainingsjahres strebt man zum Wettkampf die maximale Leistungsfähigkeit an. Typischerweise setzt sich ein Trainingsjahr aus Vorbereitungsperiode, Aufbauperiode und Übergangsperiode zusammen.\cite[279]{Trainingswissenschaft} Die Vorbereitungsphase umfasst die Grundlagenausdauer und allgemeine Fitness. Im Radsport liegt diese im Allgemeinen in der Wintersaison. Innerhalb einer Aufbauperiode werden zunehmend die wettkampfsspezifischen Fähigkeiten ausgebaut. Das Training wird auf die individuellen Anforderungen eines Wettkampfs abgestimmt. Dabei kann abhängig von Länge und Anzahl der Wettkampfsphasen einfach oder mehrfach periodisiert werden.
\subsection{Zyklisierung}
    Das Training gliedert sich in verschiedene Zyklen mit hierarchischem Aufbau. Sie geben dabei die Belastung und Regenerationsphasen im Trainingsplan vor. Übergeordnete Stufen wirken auf darunterliegende Zyklen ein, indem sie Trainingsumfang, Methoden und auch die Wahl der Leistungsbereiche beeinflussen. \cite[283]{Trainingswissenschaft}
    Makrozyklen haben eine Dauer von 4-5 Monaten. Sie bestehen aus Mesozyklen, die eine Dauer von circa 4 Wochen haben. Jede Woche wird dabei als Mikrozyklus bezeichnet. Darin werden Trainingseinheiten geplant. Sogenannte strukturierte Trainingseinheiten definieren zusätzlich die enthaltenen Trainingsabschnitte. Maßgeblich für die Unterteilung einer Einheit ist dabei die Trainingsmethode \ref{grundlagen:methoden}.
\subsection{progressive Belastungssteigerung}
    Nach einem Trainingsreiz über der Reizschwelle reagiert der Körper mit Anpassungen. Die Leistung wird gesteigert, indem die Reizschwelle erhöht wird. Weitere Leistungssteigerungen durch Anpassungen des Körpers treten erst ein, wenn die veränderte Reizschwelle übertroffen wird. Die Belastung durch Trainingseinheiten erzielt einen optimale Leistungssteigerung, wenn sie progressiv gestaltet ist. Jedoch schädigt ein zu starker überschwelliger Reiz das System. Bei unterschwelligen Reizen führen die Trainingseinheiten nicht zur Steigerung der Leistung, da keine körperliche Anpassung stattfindet. \cite[58]{Seidenspinner2005} Progression kann dabei durch Steigerung der wöchentlichen Trainingstage, der Dauer der Einheiten oder der Trainingsintensität erreicht werden. Möglich ist auch die Verkürzung der Pausen.\newline 
    Mesozyklen und Makrozyklen sind je mit steigender Belastung geplant. Der Trainingsumfang dient hier als Richtwert. \cite[60-61]{Radsporttraining}
\subsection{Regeneration}
    Die Anpassung des Körpers an neue überschwellige Reize erfolgt nicht beim Training selbst, sondern in der Regenerationszeit. Um Übertraining zu verhindern ist mindestens ein Regenerationstag in der Woche einzuplanen. 
    Zur optimalen Leistungssteigerung werden sowohl Trainingseinheiten als auch Belastungspausen zwischen den Einheiten eingeplant. Die Länge der Regenerationspause ist dabei abhängig von Intensität und Dauer der vorangegangenen Leistung. Auch Alter, Geschlecht, Ernährung oder auch Tagesform spielen eine Rolle. Trotzdem verlieren einige Richtlinien nicht die Allgemeingültigkeit. Beispielsweise folgt nach Wettkämpfen eine Phase der Regeneration. Diese schließt auch aktive Regeneration ein, die im Leistungsbereich der Kompensation stattfindet.
\subsection{Superkompensationsmethode}
    Die Superkompensation bezeichnet die gesteigerte Leistungsfähigkeit, bei optimaler Zeitplanung der Regeneration nach einer Belastung.\cite[163]{Trainingswissenschaft} Dabei ist eine Übertraining aber auch Unterforderung zu vermeiden um eine bestmögliche Belastung zu erreichen. In der Erholungsphase gibt es einen Zeitpunkt an dem erhöhte Leistungsbereitschaft besteht. Der Zeitpunkt ist abhängig von Intensität und Umfang der Einheit aber auch von persönlichen Voraussetzungen wie Leistungsstand und Erholungsfähigkeit. \newline
    Folgende Richtlinien ergeben sich für die Trainingsplanung. \cite[44-46]{Radsporttraining} Nur ein Training pro Tag wird eingeplant bei mehrstündigen Ausdauerbelastungen. Die Belastung sollte dabei im Block erfolgen -- circa drei bis fünf aufeinanderfolgende Tage bei Einheiten für die Grundlagenausdauer. Bei Einheiten mit hoher Intensität sind die Blöcke kürzer gestaltet. Im Anschluss an einen Belastungsblock folgt ein Regenerationstag. An Erholungstagen ist keine Trainingseinheit oder maximal eine Belastung im Regenerationsbereich erlaubt. Einheiten im Bereich des Krafttrainings werden bevorzugt nach einem Regenerationstag eingeplant. 
    \TODO{Nicht benutzt? \cite[60]{Radsporttraining}}

\section{Trainingseinheit}
\label{grundlagen:einheiten}
Sind Trainingspläne monoton gestaltet, kann es zu Übertraining, Leistungsstagnation oder Krankheit kommen. Diese werden unterbunden durch strukturierte Trainingseinheiten, die in verschiedene Trainingsabschnitte geteilt werden. Dabei variieren sie in der Trainingsmethode und den enthaltenen Leistungsbereichen.
\subsection{Trainingsmethoden}
\label{grundlagen:methoden}
    Der Aufbau einer Trainingseinheit wird durch die Trainingsmethode bestimmt. Im folgenden handelt es sich lediglich um eine Auswahl der möglichen Trainingsmethoden für Radsportler in der Aufbauphase.  \cite[40-43]{Radsporttraining} \TODO{reicht es hier zu zitieren?}
    \subsubsection{Dauerleistungsmethode DL}
    Bei der Dauerleistungsmethode wird die gesamte Trainingseinheit in einem Leistungsbereich ausgeführt. Oft findet diese in niedrigeren Intensitäten Verwendung. Ein Großteil der Vorbereitungsphase fokussiert das Training auf den Grundlagenausdauerbereich, der mit der Dauerleistungsmethode trainiert wird.
    \subsubsection{Fahrtspiel FS}
    Das Fahrtspiel ist eine spezielle Art der Dauerleistungsmethode. Dabei gibt es keine Vorgaben hinsichtlich der Intensität und des Leistungsbereichs. Die Belastung erfolgt ohne Pause, kann aber mehrere Leistungsbereiche ansprechen. Das Training wird persönlich gesteuert oder kann durch die äußeren Gegebenheiten (bergiger Streckenverlauf) beeinflusst werden. Vorgaben über Anteile der Trainingsbereiche sind zeitlich flexibel eingebaut. Nicht jeder Leistungsbereich muss dabei abgedeckt werden. 
    \subsubsection{Intervallmethode IV}
    Intervalleinheiten alternieren Belastung und Erholungsphasen. Die Pausen reichen nicht für eine vollständige Erholung aus und können auch als aktive Pausen gestaltet werden. Serienpausen sind längere Pausen und gruppieren mehrere Belastungen. Besonders Einheiten mit hoher Intensität werden mithilfe der Intervallmethode trainiert. 
    \subsubsection{Wiederholungsmethode WH}
    Ähnlich wie in der Intervallmethode gibt es einen Wechsel zwischen Belastung und Erholung. Jedoch dienen die Pausen bei der Wiederholungsmethode der vollständigen Erholung. Anhand der Herzfrequenz kann kontrolliert werden, zu welchem Zeitpunkt die nächste Belastung startet. Auch hiermit werden tendenziell Trainingsbereiche mit hoher Intensität trainiert.

\subsection{Belastungsbereiche}
Die Aufbauphase dient zur Vorbereitung auf einen Wettkampf. Im Radsport gibt es verschiedene Disziplinen (Straßenrennen, Zeitfahrten, Bergfahrten, uvm.) mit unterschiedlichem Belastungsprofil. Um die wettkampfsspezifischen Anforderungen zu trainieren, werden die Leistungsbereiche des Radsports in dessen Abhängigkeit gewichtet. Maßgeblich ist die Intensität der Belastung. Folgende Einteilung der Bereiche findet man in Lindners Radsporttraining. \cite[31-39]{Radsporttraining} Obwohl die Benennung nicht übereinstimmt, wurde in vergleichbarer Literatur \cite[27]{Ausdauertrainer} eine analoge Einteilung in sieben Stufen vorgenommen.
\subsubsection{Kompensationsbereich (KB)}
Der Leistungsbereich mit niedrigster Intensität ist der Kompensationsbereich. Dieses Training belastet mit 60\% der maximalen Herzfrequenz. Es wird eingesetzt zur aktiven Regeneration oder Kompensation. Üblicherweise folgt ein Kompensationstraining nach sehr intensiven Einheiten oder Wettkampfsbelastungen. Beispiele für ein Kompensationstraining sind Ausfahrten oder Läufe mit geringer Intensität, Spaziergänge oder Mobilisationstrainings. \cite[31-32]{Radsporttraining}
Eine Regenerationsfahrt hat eine maximale Länge von 2 Stunden. Das alternative Lauftraining sollte 45 Minuten nicht übersteigen und ist nur außerhalb der Rennsaison empfohlen.
\subsubsection{Grundlagenausdauer (GA)}
Trainingseinheiten mit leichter Intensität werden für die Entwicklung der Grundlagenausdauer verwendet. Sie steigern die Leistung im aeroben Bereich und trainieren so denn Fettstoffwechsel. 
Training im Grundlagenausdauerbereich ist auch für die Einheiten mit hoher Intensität von Bedeutung. Die Grundlagenausdauer nimmt einen großen Teil des Trainingsplans ein.
\subsubsection{Entwicklungsbereich (EB)}
Im Entwicklungbereich wird die wettkampfsspezifische Ausdauer im mittleren Intensitätsbereich trainiert. Zusätzlich zum Fettstoffwechsel bindet es auch den Kohlenhydratstoffwechsel ein. Es wird im aeroben-anaeroben Bereich traininert. Oft wird es als Wiederholungsmethode durchgeführt.
\subsubsection{Spitzenbereich (SB)}
Die hohe Intensität im Spitzenbereich verwendet den Kohlenhydratstoffwechel. Hier wird die anaerobe Schwelle übertroffen. Ziel ist es die Schnelligkeit und Schnelligkeitsausdauer auf die Wettkampfsbedingungen zu entwickeln. Überwiegend sind Trainingseinheiten mit der Intervallmethode strukturiert und unmittelbar vor Wettkämpfen platziert.
\subsubsection{Maximal- und Schnellkraftsbereich (K1-K2)}
Im aeroben Bereich befindet sich die Entwicklung der Maximal- und Schnellkraft. Besonders bei Strecken mit hoher Steigung und Bergfahrten gewinnt dieser Bereich an Bedeutung. Die Maximal- und Schnellkraft ermöglicht es hohe Übersetzungen auch mit hoher Tretfrequenz zu beherrschen. Häufig wird dafür die Wiederholungsmethode eingesetzt.
\subsubsection{Kraftausdauer (K3-K4-K5)}
Auch hier ist Trainingsziel eine hohe Tretfrequenz mit hoher Übersetzung zu fahren. Diese soll auch über längere Zeit gehalten werden. Die Energiebereitstellung ist im aerob-anaeroben Übergangsbereich. Bergfahrten mit Tempo- und Rythmuswechsel sind eine mögliche Trainingseinheit für diesen Leistungsbereich.

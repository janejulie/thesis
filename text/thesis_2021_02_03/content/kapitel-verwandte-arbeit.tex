\chapter{Verwandte Arbeiten}
\label{sec:verwandt}

\section{Trainingsoptimierung für den Laufsport}
\label{sec:verwandt:sec1}
\TODO{Wird mir noch geschickt}
Analog zu dieser Arbeit zum Radsport wurde bereits die Trainingsplanerstellung für das Laufen in einer vorherigen Bachelorarbeit optimiert. Besonders in der Struktur der Anwendung sowie dem Umfang der Arbeit gibt es Parallelen. Viele trainingswissenschaftliche Grundlagen gelten unabhängig der Sportart. Darunter fallen die Periodisierung, Zyklisierung und das Prinzip der Superkompensation. Unterschiede gibt es in den einzelnen Wettkampfsarten sowie den zu trainierenden Leistungsbereichen und konkreten Trainingseinheiten.

\section{Verwandte Systeme}
Es gibt bereits Programme zur Trainingssteuerung und damit auch zur Planung von Einheiten. Einige bestehenden Systeme werden unter den Kriterien des Personalisierungsgrades, der Zielgruppe, der Komplexität und des Funktionsumfangs sowie den Kosten betrachtet.

\subsection{Strava}
Strava \cite{StravaWeb} ist besonders unter im Laufsport und Radsport eine weitverbreitete Anwendung zur Trainingssteuerung. Viele Anwender:innen sind im Freizeit- und Amateursport angesiedelt. In erster Linie wird Strava zur Trainingsdokumentation verwendet. Das soziale Netzwerk ermöglicht den Nutzern und Nutzerinnen das Teilen ihrer sportlichen Aktivitäten. Trainingseinheiten können über das Smartphone oder verbundene Geräte auch aufgezeichnet werden. Mit einem kostenpflichtige Abonnement erhält der Nutzer eine Auswahl von zehn Trainingsplänen, die jedoch keine Individualisierung beinhalten. Es kann lediglich zwischen den vorgefertigten Plänen ausgewählt werden. Strava hat unter Athleten zwar eine große Reichweite, der Fokus liegt jedoch bei der Trainingsdokumentation und dem Teilen der Aktivitäten im sozialen Netzwerk statt der Planung einzelner Trainingseinheiten.

\subsection{PerfectPace}
Ein umfassendes Tool zur Trainingssteuerung ist PerfectPace.\cite{PerfectPaceWeb} Dabei richtet sich die Anwendung vorwiegend an die Vorbereitung auf einen Triathlon und deckt das Laufen, Radfahren und Schwimmen ab. Der ganzheitliche Trainingsplan passt sich dynamisch an. Die Grundlage ist dabei ein KI Modell. Durch die anfallenden Kosten und die spezialisierte Ausrichtung, eignet sich die Anwendung nur bedingt für ambitionierten Freizeitradsportler/innen. Diese Anwendung addressiert in erster Linie Amateur Triathleten und Triathletinnen.

\subsection{2PEAK}
Ein weiteres kostenpflichtiges Programm ist 2PEAK.\cite{2PeakWeb} Es bietet eine umfassende Trainingssteuerung und stellt sowohl die Trainingsplanerstellung, als auch die dynamische Anpassung an Trainingsaktivitäten zur Verfügung. Dabei ist Trainingsdokumentation direkt in der Anwendung möglich. Auch wenn die Dokumentation des Trainings nötig ist für die dynamische Anpassung des Trainingsplans, falls Trainingseinheiten nicht nach der Planung durchgeführt werden. Damit einhergehend ist jedoch auch ein erhöhter Planungs- und Dokumentationsaufwand. Da ein Abonnement abgeschlossen werden muss, fallen Kosten an.

% \subsection{TrainingPeaks}
% \cite{TrainingPeaksWeb}
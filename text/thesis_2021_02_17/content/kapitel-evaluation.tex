\chapter{Evaluation}
\label{sec:evaluation}
An der Zielgruppe orientiert, werden zur Beurteilung zwei Beispiele herangezogen.

\section{Anwendungsfall Freizeitsport}
Im ersten Szenario ist die Modellierung mit einem geringen Umfang durchgeführt worden. Der Trainingsplan umfasst Einheiten für drei Monate. Bei der Wettkampfsdisziplin handelt es sich um ... . 

\section{Anwendungsfall Amateursport}
Mit fünfmonatiger Laufzeit und einem hohen Trainingsumfang, ist dieser Plan für Anforderungen aus dem Amateursport erstellt. Die Wettkampfsdisziplin .

\section{Evaluation der Individualität}
Die Eingaben des Benutzers sind in fünf Faktoren fetsgehalten. Das Wettkampfsdatum ist dabei bei der modellierung unerheblich. Erst bei der Visualisierung der Trainingseinheiten kommt es ins Spiel. Die anderen vier Eingaben ermöglichen dem Benutzer eine Abstimmung auf die persönlichen Trainingsziele. 

\section{Evaluation der Anwendbarkeit}
Kürzerer Plan hat nicht soviel Spielraum in der Verteilung also höhere Abweichung. Nicht so schlimm, da gering genug und auch weniger nah an Professionalisierung.
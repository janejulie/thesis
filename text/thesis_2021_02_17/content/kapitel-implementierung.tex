\chapter{Implementierung}
\label{sec:implementierung} 
Bei der Implementierung der Modellierung wird die objektorientierte Programmiersprache JAVA \cite{java} verwendet. Nativ bietet sie keine Constraint Programmierung an, aber diesbezüglich wird auf choco-solver \cite{ChocoSolverWeb} zurückgegriffen. Mit dieser Open-Source Bibliothek ist die Modellierung von persönlichen und Lehrprojekten möglich. \par
Aus der Hierarchie der Zyklen lassen sich die Objektklassen entwerfen. Auch wenn die übergeordnete Instanz der Makrozyklus ist, erfolgt die Optimierung erst auf Ebene des Mesozyklus \hyperref[sec:modellierung:model]{(siehe Kapitel \ref{sec:modellierung:model})}. So wird jeder Monat unabhängig der Anderen modelliert und der Einsatz des choco-solvers auf die \texttt{Meso}-Klasse beschränkt. Gesteuert wird die Gewichtung der Leistungsbereiche in den einzelnen Monaten durch zwei Faktoren. Die Länge des Plans (3, 4 oder 5 Monate) bestimmt die Anzahl der Meso-Instanzen im Makrozyklus. Des Weiteren steigt bei Mesozyklen die Gewichtung der wettkampfsspezifischen Ausdauer bei Näherung an den Wettkampfstermin.
Dennoch ist der Solver in ein Programm eingebettet, dass bereits die Eingabe und Ausgabe des Benutzers handhabt. Die Interaktion mit dem Programm ist so unabhängig von der Modellierung. \par

\section{Eingaben}
Um den Trainingsplan zu individualisieren erfasst das Programm die Eingaben des Nutzers. Die Wettkampfsdisziplin korreliert mit dem Trainingsziel. An der ausgewählten Disziplin macht sich dann die Gewichtung der Leistungsbereiche fest. Der wöchentliche Trainingsumfang des Plans limitiert die Trainingszeit pro Woche. Die Anzahl der Stunden lässt auch Rückschlüsse auf die Professionalität zu. Während im Profibereich der Trainingsumfang über 20 Stunden beträgt, unterschreiten Amateure diesen Wert üblicherweise. Bei einem Wert bis zu 5 Stunden, spricht man oft von Freizeitsport. \TODO{Belegen} \newline
Nicht nur die wöchentlichen Stunden, sondern auch die wöchentlichen Trainingstage werden bei der Erstellung des Plans berücksichtigt. Die Anzahl der Tage steuert die Häufigkeit der Einheiten.
Wie in \ref{grundlagen:sport:belastungsbereiche} aufgeführt, umfasst ein Training mehrere Belastungsbereiche. Weitestgehend werden die Konstanten der Modellierung über die Eingabe erfasst. 
\begin{itemize}
    \item Ziel/Disziplin: Straßeneinzelrennen, Rundstrecke, Bergzeitfahrt
    \item wöchentlicher Trainingsumfang: Anzahl Stunden
    \item wöchentliche Trainingstage: Anzahl Tage
    \item Wettkampfstermin: Datum
    \item Dauer des Plans: 3/4/5 Monate
\end{itemize}

\section{UML-Diagramm der Modellierung}
\begin{figure}[h]
    \begin{tikzpicture}
        \umlclass[y=5, fill=white, type = abstract]{Macro}{
            - sessions : ArrayList<Session> \\
            - numMonth : int \\
            - maxTrainingMinutes : int \\
            - maxTrainingDays : int \\
            - compDay : Date \\
            - ranges : HashMap<Range, Double> \\
        }{
            + solvePlan() : void \\
            \textit{+ setRanges() : void} \\
            + validateRanges() : void \\
        }
        
        \umlclass[x=8, y=5, fill=white]{Meso}{
            - model : Model \\
            - plan : Solution \\
            - startDay : Date \\
            - targetRanges : int[6] \\
            - targetMinutes : int[4] \\
            - name : IntVar[] \\
            - minutes : IntVar[] \\
            - method : IntVar[] \\
            - ranges : IntVar[][] \\    
            - distanceRanges : IntVar[] \\
            - overallDistance : IntVar \\

        }{
            - initializeModel() : void \\
            - defineConstraints() : void \\
            + solveMonth() : void \\
            + getPlan() : Solution \\
        }
        \umlclass[x=8, y=-3, fill=white]{Session}{
            - name : String \\
            - minutes : int \\
            - distribution : HashMap<Range, Integer> \\
            - day : LocalDate \\
            - method : Method \\
        }{}
        \umlclass[x=1, y=0, fill=white]{Strasseneinzel}{}{+ setRanges() : void}
        \umlclass[x=1, y=-2, fill=white]{Rundfahrt}{}{+ setRanges() : void}
        \umlclass[x=1, y=-4, fill=white]{Bergfahrt}{}{+ setRanges() : void}
        
        \umlHVinherit[anchor2=-130]{Strasseneinzel}{Macro}
        \umlHVinherit[anchor2=-130]{Rundfahrt}{Macro}
        \umlHVinherit[anchor2=-130]{Bergfahrt}{Macro}
        \umluniassoc[arg=-mesos ,  mult2=3..5, pos =0.95, align=right]{Macro}{Meso}
        \umluniassoc[arg=-sessions , mult2=28, pos =0.80, align=right]{Meso}{Session}
    \end{tikzpicture}
    \caption{UML Diagram der Modellierung}
    \label{fig:uml:solver}
\end{figure}

\subsection{Macro-Klasse}
Diese Klasse koordiniert das Erstellen der Mesozyklen und ist der Einstiegspunkt für Operationen auf den Trainingsplan. Je nach Planlänge, wird für jeden Monat eine Mesoinstanz generiert. Hierfür werden aus dem Prinzip der Periodiesierung und progressiven Belastung bereits die Zielwerte in den einzelnen Wochen berechnet, um sie an die Mesozyklen weiterzugeben.
Die Lösungsinstanzen der Mesozyklen ergeben dann zusammengesetzt den Trainingsplan. Das ist analog zur Hierarchie der Zyklen gestaltet. \newline
Über Vererbung werden die verschiedenen Wettkampfsdisziplinen realisiert. Die Hook-Operation \texttt{setRanges()} definiert für jeden Belastungsbereich die Gewichtung. Mit \texttt{validateRanges()} wird sichergestellt, dass diese Verteilung zu 1 summiert. 
Um Laufzeit zu optimieren wird die Optimierung der einzelnen Mesoinstanzen parallelisiert angestoßen.

\subsection{Meso-Klasse}
Gekapselt in eine Klasse wird hier die Modellierung von 28 Tagen vorgenommen. Zum Einsatz kommt hier der choko-solver. Für die Variablen werden die zugehörigen IntVar-Klassen angelegt. \TODO{Hier den Code einfügen?} 
Nach dem Lösungsprozess erstellt die Klasse aus den Werten die passenden Sessionobjekte für jeden der 28 Tage.
Um die Modellierung in angemessener Zeit zu ermöglichen wurde mithilfe des choko-Solvers eine zeitliche Begrenzung von 30 Sekunden für den Löäsungsprozess festgelegt. So kann der Lösungsprozess besser gesteuert werden. Durch die Diskretisierung der Belastungsbereiche ist die exakte Vorgabe an Minuten eventuell gar nicht zu treffen. Die genauer Evaluation der erstellten Trainingspläne erfolgt in \hyperref[sec:evaluation]{Kapitel \ref{sec:evaluation}}.

\subsection{Session-Klasse}
Die Sessionobjekte vereinfachen die Visualisierung. Die enthalten alle charakteristischen Daten eines Trainingstages auf die in \ref{sec:modellierung:output} eingegangen wird/wurde. 

\section{Modularisierung}
Die Grundlage dieser Arbeit war eine vorangegangene Bachelorarbeit, die der Laufsport betrifft. Mit Ausblick auf die Erweiterung um das Schwimmtraining, ist durch Kombination dieser Arbeiten vorstellbar zu einem Trainingsplan für Triathleten. 
Aus diesem Grund ist die Arbeit modular gegliedert.
Für viele Sportarten gelten die Trainingsprinzipien der Zyklisierung, Periodisierung, progressive Belastung und Superkompensation. Diese Struktur kann für andere Ausdauersportarten übernommen werden. Um die Trainingsplanerstellung für andere Sportarten zu modellieren, sind folgende Erweiterungen im Code nötig. \par
Die Definition der Leistungsbereiche, Trainingsmethoden und validen Trainingseinheiten ist über Aufzählungstypen (engl. enumeration) erfolgt. Diese enthalten spiegeln eine endliche Wertemenge wieder.\par
\begin{figure}[h]
    \centering
    \begin{tikzpicture}
        \umlclass[type=enumeration, fill=white]{Range}{
            Kompensation   \\ 
            Grundlagenausdauer   \\ 
            Entwicklungsbereich   \\ 
            Spitzenbereich   \\ 
            Kraftausdauer1   \\ 
            Kraftausdauer4   
        }{}
        
        \umlclass[type=enumeration, x=4.25, fill=white]{Method}{
            Pause \\
            Dauerleistung   \\ 
            Fahrtspiel  \\
            Intervall \\
            Wiederholung  \\
        }{}
        \umlclass[type=enumeration, x=9, fill=white]{SessionPool}{
            Pause\\
            Kompensationstraining \\
            ExtensiveFahrt   \\ 
            Fettstoffwechsel  \\
            IntensiveFahrt \\
            ExtensiveKraftausdauerfahrt  \\
            Einzelzeitfahrt  \\
            ExtensivesFahrtspiel  \\
            IntensivesFahrtspiel  \\
            IntensiveKraftausdauer  \\
            Schnelligkeitsausdauer  \\
            Sprinttraining  \\
        }{
            getPause() \\
            getDL() \\
            getFS() \\
            getIV() \\
            getWH() \\
        }
    \end{tikzpicture}    
    \caption{Enumeration um sportartspezifisches zu Kapseln}
    \label{fig:uml:enumeration}
\end{figure}
Die Erweiterung des Modells um weitere Leistungsbereiche in Range ist möglich, aber erfordert im SessionPool die Festlegung der Zeitspannen für diesen Belastungsbereich. Eher ist davon auszugehen, dass nach der Festlegung für eine Sportart die Belastungsbereiche final sind. Vorteil der Kapselung ist hier, dass eine neue Kombination dieser drei Aufzählungstypen genutzt werden kann um mit dem Modell andere Ausdauersportarten zu lösen. Denn die konkreten Belastungsbereiche beeinflussen die Modellierung nicht. Diese ist unabhängig von sportartspezifischen Ausprägung der Trainingseinheiten implementiert. Die Liste der Trainingsmethoden dient zur Zuweisung der Trainingseinheiten zu ihrer Trainingsmethode. Im SessionPool sind die Trainingseinheiten definiert mit ihren möglichen Zeitspannen je Belastungsbereich. Neue Arten von Trainingseinheiten können also mit geringem Aufwand hinzugefügt werden. Das Ändern der Modellierungsklasse ist dafür nicht erforderlich.

\section{Ausgabe}
\label{sec:modellierung:output}
Die Ausgabe des Plans ist über zwei Wege verfügbar. In der Implementierung ist eine grafische Benutzungsoberfläche zur tabellarischen Ansicht der Trainingseinheiten inklusive. Über die grafische Schnittstelle erfolgen die Eingaben und nach Erstellung des Plans auch die Möglichkeit diesen als PDF-Dokument abzuspeichern. Die Trainingseinheit wird definiert durch die nachfolgenden Parameter.
\begin{figure}[h]
    \begin{tikzpicture}
        \umlclass[y=5, fill=white]{Main}{
            - plan : Macro
        }{
            + monitorStats() : void \\
            + createPlan() : void \\
            + createTable(i : int) : void \\
            + createPDF(i : int) : void \\}
        \umlclass[x=9, y=5, fill=white]{OutputTrainingTable}{
        }{
            + displayPlan() : void \\
        }
        \umlVHVassoc[]{Main}{OutputTrainingTable}
    \end{tikzpicture}    
    \caption{UML-Diagram für die Interaktion mit dem Modell}
    \label{fig:uml:solver}
\end{figure}
\begin{itemize}
    \item Tag: Datum
    \item Dauer: Anzahl Stunden
    \item Trainingsarten: TrainingPool \ref{anhang:trainingsarten}
    \item Trainingsbereich: Range \ref{grundlagen:sport:belastungsbereiche}
    \item Trainingsmethode: Method \ref{grundlagen:methoden}
\end{itemize}

Das UML-Diagramm des vollständigen Programms ist im Anhang unter \ref{anhang:uml} aufgeführt und besteht aus der Zusammensetzung der obigen Teile.

% ALTERNATIVSÄTZE
% Außerdem besteht die Möglichkeit den Trainingsplan als einfaches PDF-Dokument herunterzuladen.
% Darstellung der Trainingseinheiten als Liste in der Java Applikation
% Die konkrete Umsetzung des Softwareprojekts wird als Java-Anwendung zur Verfügung gestellt.
% Alle oben genannten Funktionen werden unterstützt.Bei der Lösung des Optimierungsproblems kommt das Framework choco-solver zum Einsatz. Die Optimierung wird ausgelöst auf jeden Monat.
% Die Berechnung einer Lösunginstanz der Mesozyklen wird parallelisiert ausgeführt.

% Übertragbarkeit auf andere Sportarten? Welche Constraints sind speziell für den Radsport vs. allgemeine Trainingsprinzipien
% modulare Entwicklung der Anwendung?
% Einbettung des Solvers in Java Programm
% Time Limits oder andere Arten zur Steuerung des Solvers -> Kommt dann in Implementierung
% ? Trainingsalternativen = Auswahl an möglichen Einheiten
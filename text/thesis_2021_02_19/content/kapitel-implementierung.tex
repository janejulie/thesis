\chapter{Implementierung}
\label{sec:implementierung} 
Bei der Implementierung der Modellierung wird die objektorientierte Programmiersprache Java \cite{java} verwendet. Nativ bietet sie keine Constraint Programmierung an, aber diesbezüglich wird auf Choco Solver \cite{ChocoSolverWeb} zurückgegriffen. Mit dieser Open-Source Bibliothek ist die Modellierung von persönlichen und Lehrprojekten möglich. \par
Aus der Hierarchie der Zyklen wie in \ref{abbildung:modellierung:schema} lassen sich die Objektklassen entwerfen. Auch wenn die übergeordnete Instanz der Makrozyklus ist, erfolgt die Optimierung erst auf der Ebene des Mesozyklus. So wird jeder Monat unabhängig der Anderen modelliert und der Einsatz des Choco Solvers auf die \texttt{Meso}-Klasse beschränkt. Gesteuert wird die Gewichtung der Belastungsbereiche in den einzelnen Monaten durch zwei Faktoren: Die Länge des Plans (3, 4 oder 5 Monate) bestimmt die Anzahl der Meso-Instanzen im Makrozyklus. Des Weiteren steigt bei Mesozyklen die Gewichtung der wettkampfspezifischen Ausdauer bei Näherung an den Wettkampftermin. Diese Vorgaben werden bereits im Makrozyklus berechnet und dann an die Meso-Instanzen weitergegeben.
Zusätzlich ist die Modellierung in ein Programm eingebettet, das bereits die Eingabe und Ausgabe handhabt. Die Interaktion mit dem Programm ist so unabhängig von der Trainingsplanerstellung. \par

\section{Eingaben}
Um den Trainingsplan zu individualisieren, erfasst das Programm die Eingaben über eine grafische Oberfläche. Die Wettkampfdisziplin korreliert mit dem Trainingsziel. An der ausgewählten Disziplin macht sich die Gewichtung der Belastungsbereiche fest. Der wöchentliche Trainingsumfang des Plans limitiert die Trainingszeit pro Woche. Über die Anzahl der Stunden lassen sich Rückschlüsse auf die Professionalität des Trainings ziehen. Während im Profibereich der Trainingsumfang über 20 Stunden beträgt, unterschreiten Amateure diesen Wert üblicherweise. Bei einem Wert bis zu fünf Stunden pro Woche, spricht man oft von Freizeitsport. \newline
Nicht nur die Wochenstunden, sondern auch die wöchentlichen Trainingstage werden bei der Erstellung des Plans berücksichtigt. Die Anzahl der Tage steuert die Häufigkeit der Einheiten.
Wie in \ref{grundlagen:sport:belastungsbereiche} aufgeführt, umfasst ein Training mehrere Belastungsbereiche. Weitestgehend werden die Konstanten der Modellierung über die Eingabe erfasst. 
\begin{itemize}
    \item Ziel/Disziplin: Straßeneinzelrennen, Rundstrecke, Bergzeitfahrt
    \item wöchentlicher Trainingsumfang: Anzahl Stunden 
    \item wöchentliche Trainingstage: Anzahl Tage 3-6
    \item Wettkampfstermin: Datum
    \item Dauer des Plans: Monate 3-5
\end{itemize}
    
\section{Klassendiagramm der Modellierung}
\begin{figure}[h]
    \begin{tikzpicture}
        \umlclass[y=5, fill=white, type = abstract]{Macro}{
            - sessions : ArrayList<Session> \\
            - numMonth : int \\
            - maxTrainingMinutes : int \\
            - maxTrainingDays : int \\
            - compDay : Date \\
            - ranges : HashMap<Range, Double> \\
        }{
            + solvePlan() : void \\
            \umlvirt{+ setRanges() : void} \\
            + validateRanges() : void \\
        }
        
        \umlclass[x=8, y=5, fill=white]{Meso}{
            - model : Model \\
            - plan : Solution \\
            - startDay : Date \\
            - targetRanges : int[6] \\
            - targetMinutes : int[4] \\
            - name : IntVar[] \\
            - minutes : IntVar[] \\
            - method : IntVar[] \\
            - ranges : IntVar[][] \\    
            - distanceRanges : IntVar[] \\
            - overallDistance : IntVar \\

        }{
            - initializeModel() : void \\
            - defineConstraints() : void \\
            + solveMonth() : void \\
            + getSessions() : Session[] \\
            + getPlan() : Solution \\
        }
        \umlclass[x=8, y=-3, fill=white]{Session}{
            - name : String \\
            - minutes : int \\
            - distribution : HashMap<Range, Integer> \\
            - day : LocalDate \\
            - method : Method \\
        }{
        }
        \umlclass[x=1, y=0, fill=white]{Strasseneinzel}{}{+ setRanges() : void}
        \umlclass[x=1, y=-2, fill=white]{Rundfahrt}{}{+ setRanges() : void}
        \umlclass[x=1, y=-4, fill=white]{Bergfahrt}{}{+ setRanges() : void}
        
        \umlHVinherit[anchor2=-130]{Strasseneinzel}{Macro}
        \umlHVinherit[anchor2=-130]{Rundfahrt}{Macro}
        \umlHVinherit[anchor2=-130]{Bergfahrt}{Macro}
        \umluniassoc[arg=-mesos ,  mult2=3..5, pos =0.95, align=right]{Macro}{Meso}
        \umluniassoc[arg=-sessions , mult2=28, pos =0.80, align=right]{Meso}{Session}
    \end{tikzpicture}
    \caption{Klassendiagramm der Modellierung}
    \label{fig:uml:solver}
\end{figure}

\subsection{\texttt{Macro}-Klasse}
Diese Klasse koordiniert das Erstellen der Mesozyklen und ist der Einstiegspunkt für Operationen auf dem Trainingsplan. Je nach Planlänge wird für jeden Monat eine Mesoinstanz generiert. Hierfür werden aus dem Prinzip der Periodisierung und progressiven Belastung bereits die Zielwerte in den einzelnen Wochen berechnet und an die Mesozyklen weitergegeben. Die Ziele werden in Minuten angegeben. Die Diskretisierung der wöchentlichen Umfänge ist hier berücksichtigt worden, indem die Werte entsprechend gerundet werden.\newline
Über Vererbung werden die verschiedenen Wettkampfsdisziplinen realisiert. Die Hook-Operation \texttt{setRanges()} definiert für jeden Belastungsbereich die Gewichtung. Mit \texttt{validateRanges()} wird sichergestellt, dass diese prozentuale Verteilung zu 1 summiert. Auch das Ergebnis der Zielminuten wird auf eine Genauigkeit von fünf Minuten gerundet.
Durch die unabhängige Planung der Mesozyklen kann die Laufzeit optimiert werden, indem die Lösung der einzelnen Mesoinstanzen parallelisiert erfolgt. Die Ergebnisse aus den verschiedenen Prozessen werden im Anschluss dann zu einem Trainingsplan zusammengetragen. 

\subsection{\texttt{Meso}-Klasse}

Gekapselt in eine Klasse wird hier die Modellierung von 28 Tagen vorgenommen. Hier kommt der Choko Solver zum Einsatz. Für die Variablen werden die zugehörigen IntVar-Klassen angelegt. Die Constraints der Modellierung werden definiert. Mit \texttt{solveMonth()} wird die Lösungssuche angestoßen.
Aus den Werten der Lösungsinstanz erstellt die Klasse die passenden Session-Objekte für jeden der 28 Tage.
Um die Modellierung in angemessener Zeit zu ermöglichen wurde mithilfe des Choko Solvers eine zeitliche Begrenzung von fünf Sekunden für den Lösungsprozess festgelegt. So kann der Lösungsprozess besser gesteuert werden. Die genaue Evaluation der erstellten Trainingspläne erfolgt in \hyperref[sec:evaluation]{Kapitel \ref{sec:evaluation}}.

\subsection{\texttt{Session}-Klasse}
Die Session-Objekte vereinfachen die Visualisierung der Trainingseinheiten. Sie enthalten alle charakteristischen Daten eines Trainingstages wie er in \ref{sec:modellierung:output} ausgeführt. Sie werden erst nach der Lösung des Mesozyklus erstellt und sind über \texttt{getSessions()} abrufbar.

\section{Modularisierung}
Die Grundlage dieser Arbeit war eine vorangegangene Bachelorarbeit, die den Laufsport betrifft. Mit Ausblick auf die Erweiterung um das Schwimmtraining, ist durch Kombination der Arbeiten vorstellbar, die Trainingsplanerstellung für Triathletinnen und Triathleten zu optimieren. 
Aus diesem Grund ist die Arbeit modular gegliedert.
Für viele Sportarten gelten die Trainingsprinzipien der Zyklisierung, Periodisierung, progressiven Belastung und Superkompensation. Diese Struktur kann besonders für andere Ausdauersportarten übernommen werden.\par
Die Definition der Belastungsbereiche, Trainingsmethoden und validen Trainingseinheiten ist über Aufzählungstypen (engl. enumeration) erfolgt. Diese spiegeln die endliche Wertemenge der Variablen wieder.\par
\begin{figure}[h]
    \centering
    \begin{tikzpicture}
        \umlclass[type=enumeration, fill=white]{Range}{
            Kompensation   \\ 
            Grundlagenausdauer   \\ 
            Entwicklungsbereich   \\ 
            Spitzenbereich   \\ 
            Kraftausdauer1   \\ 
            Kraftausdauer4   
        }{}
        
        \umlclass[type=enumeration, x=4.25, fill=white]{Method}{
            Pause \\
            Dauerleistung   \\ 
            Fahrtspiel  \\
            Intervall \\
            Wiederholung  \\
        }{}
        \umlclass[type=enumeration, x=9, fill=white]{SessionPool}{
            Pause\\
            Kompensationstraining \\
            ExtensiveFahrt   \\ 
            Fettstoffwechsel  \\
            IntensiveFahrt \\
            ExtensiveKraftausdauerfahrt  \\
            Einzelzeitfahrt  \\
            ExtensivesFahrtspiel  \\
            FreiesFahrtspiel  \\
            IntensiveKraftausdauer  \\
            Schnelligkeitsausdauer  \\
            Sprinttraining  \\
        }{
            getPause() \\
            getDL() \\
            getFS() \\
            getIV() \\
            getWH() \\
        }
    \end{tikzpicture}    
    \caption{Enumerations um sportartspezifisches zu Kapseln}
    \label{fig:uml:enumeration}
\end{figure}
Die Erweiterung des Modells in \texttt{Range} um weitere Belastungsbereiche ist möglich, aber erfordert im \texttt{SessionPool} die Festlegung der Zeitspannen für jede Trainingsart. Eher ist davon auszugehen, dass nach der Festlegung für eine Sportart die Belastungsbereiche fest sind und stattdessen SessionPool erweitert wird. \par 
Der Vorteil der Kapselung ist hier, dass eine neue Kombination dieser drei Aufzählungstypen genutzt werden kann um mit dem Modell andere Ausdauersportarten zu lösen. Denn die konkreten Belastungsbereiche, Methoden oder Trainingseinheiten beeinflussen die Modellierung nicht. Die Implementierung abstrahiert von sportartspezifischer Ausprägung der Trainingseinheiten. Die Liste der Trainingsmethoden dient der Zuweisung der Einheiten zu ihrer Methode. Im SessionPool sind die Trainingseinheiten mit ihren potentiell realisierbaren/umsetzbaren Zeitspannen in den verschiedenen Belastungsbereichen definiert. Dadurch können neue Arten von Trainingseinheiten auch nachträglich mit geringem Aufwand hinzugefügt werden. Die Instanz wird in SessionPool definiert und beim Abruf der Trainingsmethode an die Modellierung weitergegeben. Das Ändern der Modellierungsklasse ist dafür nicht erforderlich.

\section{Ausgabe}
\label{sec:modellierung:output}
Die Ausgabe des Plans ist über zwei Wege verfügbar. In der Implementierung ist eine grafische Benutzungsoberfläche zur tabellarischen Ansicht der Trainingseinheiten inklusive. Über die grafische Schnittstelle \ref{anhang:gui} erfolgen die Eingaben und nach Erstellung des Plans auch die Möglichkeit diesen als PDF-Dokument abzuspeichern. Die Trainingseinheit wird definiert durch die nachfolgenden Parameter.
\begin{figure}[h]
    \begin{tikzpicture}
        \umlclass[y=5, fill=white]{Main}{
            - plan : Macro
        }{
            + monitorStats() : void \\
            + createPlan() : void \\
            + createTable(i : int) : void \\
            + createPDF(i : int) : void \\}
        \umlclass[x=9, y=5, fill=white]{OutputTrainingTable}{
        }{
            + displayPlan() : void \\
        }
        \umluniassoc[arg=-table , mult2=1, pos =0.95, align=right]{Main}{OutputTrainingTable}
    \end{tikzpicture}    
    \caption{Klassendiagramm der Interaktion mit dem Programm}
    \label{fig:uml:solver}
\end{figure}
\begin{itemize}
    \item Tag: Datum
    \item Dauer: Anzahl Stunden
    \item Trainingsarten: TrainingPool \ref{anhang:trainingsarten}
    \item Trainingsbereich: Range \ref{grundlagen:sport:belastungsbereiche}
    \item Trainingsmethode: Method \ref{grundlagen:methoden}
\end{itemize}

Das UML-Diagramm des vollständigen Programms ist im Anhang unter \ref{anhang:uml} aufgeführt und besteht aus der Zusammensetzung der obigen Teile. Die \texttt{Main}-Klasse hält eine \texttt{Macro}-Instanz, die den Trainingsplan berechnet und representiert.

\section{Testfälle}
Um die Modellierung zu beurteilen, wurde eine Testklasse angelegt. Sechs Anwendungsfälle (drei Wettkampfsdisziplinen mit je geringem und hohem Umfang) dienen hier der Überprüfung. Die Implementierung ist unter \texttt{MainTest} verfügbar. \footnote{Es ist auf der Ebene der Oberfläche getestet worden, da es bei der Makroinstanz nicht möglich war die Terminierung der einzelnen Prozesse aus der Parallelisierung abzuwarten. Für diese geringe Anzahl an Tests sind die Kosten für die Erstellung der grafischen Oberfläche überschaubar.}
Für die Erstellung der Testfälle ist die frei verfügbare Java-Bibliothek JUnit verwendet worden. Die Tests überprüfen auf der Ebene der \texttt{Main}-Klasse die Existenz der Trainingseinheiten nach dem Lösungsprozess. 

% ALTERNATIVSÄTZE
% Außerdem besteht die Möglichkeit den Trainingsplan als einfaches PDF-Dokument herunterzuladen.
% Darstellung der Trainingseinheiten als Liste in der Java Applikation
% Die konkrete Umsetzung des Softwareprojekts wird als Java-Anwendung zur Verfügung gestellt.
% Alle oben genannten Funktionen werden unterstützt.Bei der Lösung des Optimierungsproblems kommt das Framework choco-solver zum Einsatz. Die Optimierung wird ausgelöst auf jeden Monat.
% Die Berechnung einer Lösunginstanz der Mesozyklen wird parallelisiert ausgeführt.

% Übertragbarkeit auf andere Sportarten? Welche Constraints sind speziell für den Radsport vs. allgemeine Trainingsprinzipien
% modulare Entwicklung der Anwendung?
% Einbettung des Solvers in Java Programm
% Time Limits oder andere Arten zur Steuerung des Solvers -> Kommt dann in Implementierung
% ? Trainingsalternativen = Auswahl an möglichen Einheiten
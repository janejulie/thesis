\chapter{Evaluation}
\label{sec:evaluation}
% An der Zielgruppe orientiert, werden zur Beurteilung zwei Beispiele von erstellten Trainingsplänen herangezogen.
Für die Evaluation der Arbeit wird ein Beispielplan betrachtet, anhand dessen die Ergebnisse der Modellierung nach Anwendbarkeit und Individualität bewertet werden.

\section{Anwendungsfall}
Der Trainingsplan ist mit einem mittleren Umfang von zehn Wochenstunden und vier Trainingstagen erstellt worden. Bei der Wettkampfdisziplin handelt es sich um ein Straßeneinzelrennen. Der vollständigen Plan ist angehängt unter \ref{anhang:beispielplan} und umfasst alle Einheiten der drei Monate.

\section{Evaluation der Anwendbarkeit}
Die Anwendbarkeit des Plans ist garantiert durch die zeitliche Limitierung des Lösungprozesses. Das bedeutet zwar, dass die Distanz zu den Belastungsbereichen nicht immer auf Null reduziert wird, bewahrt aber die Praktikabilität des Systems. Die Präzision der Optimierung wurde über die Testfälle überprüft und beträgt bei den erstellten Trainingsplänen mehr als 90\% bei Abweichungen im Vergleich zu den gesamten Trainingsminuten. Die größte Abweichung entsteht bei sehr kleinen Trainingsumfängen, da die flexible Auswahl durch die Mindestbelegung der Trainingsmethoden eingeschränkt wird. Bei einer höheren Anzahl von Trainingstagen ist die zweifache Verwendung der Trainingsmethoden weniger einschränkend in der Auswahl der Einheiten. Da bei geringerem Trainingsumfang von keiner Professionalisierung auszugehen ist, sind die höheren Abweichungen in diesem Bereich vertretbar. \par
Da die Angabe über die Stunden und Tage stark in Zusammenhang stehen, kann eine komplementäre Belegung (hohe Stunden an wenigen Tagen oder wenige Tage mit viel Umfang) verantwortlich für höhere Abweichungen sein. Hier sichert die zeitliche Begrenzung die Ausführbarkeit des Programms.

Regenerationstage besser steuern bei geringen Tagen. Folgen viele Tage ohne Trainingseinheit aufeinander, weil regelmäßige kurze Erholungsphasen einer längeren Pause vorzuziehen sind.


\section{Evaluation der Individualität}
Die Individualität des Plans ist gegeben durch die fünf Faktoren der Eingabe. Das Wettkampfdatum ist für die Modellierung zunächst unerheblich und kommt erst bei der Visualisierung der Trainingseinheiten zum Tragen. Die anderen vier Eingaben ermöglichen dem Benutzer eine Abstimmung auf die persönlichen Trainingsziele. Die Zielgruppe ist über den Umfang abgedeckt, der mit hoher Flexibilität angegeben werden kann. Die individuellen Bedürfnisse der Sportlerin oder des Sportlers können dadurch abgedeckt werden.
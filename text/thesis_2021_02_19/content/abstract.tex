\pdfbookmark[0]{Zusammenfassung}{Abstract}
\chapter*{Kurzzusammenfassung}
\label{sec:abstract}
\vspace*{-10mm}
Besonders im Amateurbereich spielt die Vorbereitung zu den Wettkämpfen eine wichtige Rolle. Die maximale Leistung auf dem Rad wird vor allem von der physischen Leistungsfähigkeit beeinflusst. Unterschiedliche Trainingsziele nehmen Einfluss auf die Trainingsinhalte eines individuellen Sportlers \TODO{gendern}. Mit der Modellierung personalisierter Trainingspläne soll ein optimaler Trainingsverlauf für eine breite Personengruppe ermöglicht werden.
Um die Leistung zu steigern, wird diese Modellierung nach Erkenntnissen der allgemeinen und radsportspezifischen Trainingswissenschaft die Trainingseinheiten aufeinander abstimmen und so strukturieren, dass alle Belastungsbereiche wettkampfsorientiert gewichtet werden.
\vspace*{20mm}

{\usekomafont{chapter}Abstract}
\label{sec:abstract-diff}

Especially in amateur sports the preperation to a competition plays a significant role. The performance on the racing bike directly correlates with the physical ability. Different road race types impact the recommended form of exercise of the sportsperson. Using personalized training plans ensures the optimal progress in training for a broad group of people.
To increase performance this model will coordinate the training sessions based on universal and bikespecific sport science and structure them in such a way, that forms of training are weighted according to the discipline of the competition.
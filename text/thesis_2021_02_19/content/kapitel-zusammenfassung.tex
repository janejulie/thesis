\chapter{Zusammenfassung}
\label{sec:zusammenfassung}
\section{Ergebnis der Arbeit}
\label{sec:zusammenfassung:ergebnis}
Die Arbeit hat gezeigt, dass die Erstellung eines Trainingsplans durch Constraint Programmierung lösbar ist. Zu diesem Zweck sind die trainingswissenschaftlichen Anforderungen an einen Trainingsplan definiert worden. Der Trainingsplan beachtet die Trainingsprinzipien der Zyklisierung, Periodisierung und progressiven Belastung. Er plant die Aufbauperiode in Abhängigkeit der angegebenen Wettkampfdisziplin. Welche Belastungsbereiche des Radsports abgedeckt werden, steht damit in einem direkten Zusammenhang. \par 
Zusammengesetzt wird ein Plan nach dem Baukastenprinzip. Die dynamisch definierten Trainingseinheiten sind Bausteine und haben verschiedene Anteile an den Belastungsbereichen. Das mathematische Modell optimiert die Auswahl der Einheiten, sodass die Anteile der Belastungsbereiche im Plan denen des Trainingsziels entsprechen. Begrenzt wird der wöchentliche Trainingsumfang in Stunden und Tagen. Auch die Variation der Trainingsmethoden nimmt hierbei Einfluss auf die Wahl. \par
Die Modellierung ist in eine eigenständige Anwendung umgesetzt worden. Die Benutzung erfolgt über eine übersichtliche grafische Benutzeroberfläche. Auf Ebene der Mesozyklen parallelisiert die Software das Lösen der einzelnen Monate. Zusammengesetzt ergeben diese den vollständigen Trainingsplan der Aufbauperiode. Durch den modularen Aufbau des Programmcodes bietet die Arbeit eine Grundlage für die Weiterentwicklung. Das Programm listet die Trainingseinheiten nach der Berechnung und ermöglicht es sie als PDF-Dokument abzuspeichern. \par
Anhand zweier beispielhafter Pläne bestätigt sich die Praktikabilität der Anwendung. Die gemessenen Abweichungen in den einzelnen Bereichen sind vertretbar, denn die Diskrepanz zum Zielwert beträgt durchschnittlich \TODO{Prozent} der Trainingsminuten. Die Anwendung terminiert in angemessener Zeit und deckt die Zielgruppe aus Freizeitsportler:innen und Amateursportler:innen erfolgreich ab.
\TODO{Überleitung}

\section{Ausblick}
\label{sec:zusammenfassung:ausblick}
Obwohl die Erstellung eines Trainingsplans erfolgreich umgesetzt wurde, gibt es Möglichkeiten zur Verbesserung und Weiterentwicklung. Auf die verschiedenen Ansätze wird nachfolgend eingegangen.

\subsection{Präzisierung}
In dieser Modellierung geben die wichtigen Grundlagen der Trainingswissenschaft die Bedingungen vor. Dennoch deckt das Modell nicht alle Details ab, die Einfluss auf die Qualität eines Trainingsplans haben. Ein exaktes System ist in der Form zwar durch die Natur der Trainingswissenschaft nicht möglich, dennoch gibt es weitere etablierte Trainingsprinzipien. \par
Durch das Aufnehmen weiterer Abhängigkeiten steigt besonders die Individualität des Plans. Mögliche Größen mit Einfluss auf den Trainingsplan sind zu untersuchen. Darunter fallen Einfluss des Alters bzw. des Trainingsalters \cite[181]{EinfuerungTrainingswissenschaft}, des Geschlechts und die Leistungsgruppe \cite[S. 173]{Radsporttraining}. Nach dem Prinzip der Superkompensation können die Trainingseinheiten besser aufeinander abgestimmt werden, in denen Blöcke einer Woche definiert werden.\par
Darüber hinaus ist sogar ein vorangestellter Leistungstest vorstellbar. Mit dieser Art der Leistungsdiagnostik erfasst man die aktuelle Leistungsfähigkeit in den einzelnen Belastungsbereichen. Defizite in einzelnen Bereichen können --  wie die Wettkampfdisziplin -- auf die Anteile der Bereiche Einfluss nehmen. Das hohe Maß an Individualität, das dadurch gewonnen wird, muss gegen den erhöhten Nutzeraufwand abgewogen werden. \par
In der Constraint Programmierung besteht die Möglichkeit Soft-Constraints zu definieren. Deren Erfüllung ist optional und bei einer Lösungsinstanz nicht immer gegeben. Falls die Randbedingung nicht erfüllt ist, wird die Instanz bei der Optimierung aber schlechter bewertet. Der Choco Solver unterstützt diesen Mechanismus nicht direkt. Jedoch erlaubt er über Reified-Constraints den Status der Erfüllbarkeit abzufragen. Addiert man die Anzahl der nicht erfüllten Bedingungen auf den Optimierungswert, rekonstruiert es die gleichen Effekte wie ein Soft-Constraint. Interessant ist dies wiederum bei Empfehlungen der Trainingsplangestaltung. Zum Beispiel rät man zu einem Erholungstag vor Einheiten, die einen großen Anteil an K123- oder K45-Belastungen beinhalten. Auch die Verteilung der Regenerationstage kann damit gleichmäßiger erfolgen. 

\subsection{Erweiterung} 
Auf der Grundlage dieser Arbeit ist es möglich die Trainingsplanerstellung weiterzuentwickeln. Aus der hierarchischen Struktur folgt die Modularisierung des Programms in Objektklassen. Denkbar sind andere Ausdauersportarten und die Ausweitung der Trainingsziele. Des Weiteren ist die Erweiterung der Makrozyklen auf die Vorbereitungsperiode und Übergangsperiode möglich. Von Vorteil ist, dass \texttt{Macro} bereits als abstrakte Klasse definiert ist. Bei der Erstellung der Mesoinstanzen definiert man die Zielwerte entsprechend der Periode anders. Weitere Änderungen an der Implementierung würden nur das Bereitstellen der Optionen über die Oberfläche betreffen. So wären die Bausteine für einen Jahresplan vorbereitet.\par

\TODO{streichen?}
Wie bereits bei der Implementierung berücksichtigt, ist es Erweiterung um andere Sportarten eine Triathleten und andere Sportarten
Zusammenschluss der Arbeiten zu einem Trainingsplan spezifisch für die Triathlon Vorbereitung möglich? \par
Trainingsdokumentation
Erweiterung zur Trainingskontrolle und Trainingsdokumentation \par
Vorgabe über Wochentag und Zeiten im Kalender -> mehr Vorgabe als nur die Trainingstage.
    
\subsection{Performance}
Diese Arbeit beschränkt die Erstellung des Plans auf maximal fünf Sekunden. Dieses Limit wurde der Modellierung hinzugefügt, um ein anwendbares System zu erhalten. Durch die zeitliche Begrenzung des Lösungsprozesses toleriert die Lösung auch eine Abweichung von den Belastungsbereichen. Die Optimierung wird vorgenommen, es ist jedoch nicht garantiert, dass die vorgegebene Lösung den optimalen Wert von Null erreicht. Obwohl die Abweichungen in einem hinreichend kleinen Rahmen ist, kann das bei der Weiterentwicklung des Programms zu Problemen führen. \newline
Eine eigene Suchstrategie kann dem entgegenwirken. Mit der Flexibilität der Fahrtspielmethode kann eine geeignete Suchstrategie die Werte so wählen dass die Distanz eliminiert wird. Eine bessere Suchstrategie würde auch die aktuelle Implementierung begünstigen, denn sie ist hängt genauso mit der Laufzeit der Lösungssuche zusammen. Sowohl eine präzisere Lösung als auch die Minderung der Wartezeit wären damit möglich.

\subsection{Zugänglichkeit}
Der Fokus dieser Arbeit liegt auf der Modellierung eines optimierten Trainingsplans, weshalb sie als eigenständige Anwendung umgesetzt ist. Um das System der Zielgruppe zur Verfügung zu stellen ist eine Webanwendung zweckmäßiger. 
Durch die Definition einer Schnittstelle mithilfe von JSON-Objekten kapselt sich die Modellierung von der Benutzerinteraktion ab. Nötig ist hierfür das Deployment der Anwendung auf einem Server. Nachfolgend wird eine Möglichkeit für eine Schnittstelle dargestellt:
\begin{minipage}{\linewidth}
\begin{lstlisting}
{ 
    "num_months": 3,
    "competition": "singleday",
    "sport": "racing bike",
    "competition_date": 22.02.2022,
    "sessions": [
        {
            date: 01.02.2022,
            minutes: 60,
            method: "intervall",
            name: "sprinttraining", 
            sections: [
                {
                    length: 45,
                    range: "GA"
                },
                {
                    length: 5,
                    range: "EB"
                },
                ...
            ]
        }
    ], 
}
\end{lstlisting}
\end{minipage}

Die grafische Oberfläche kann um eine bessere Beschreibung der Trainingseinheiten erweitert werden. Interessant ist das besonders für Einheiten mit der Intervallmethode, die dann die Länge der Intervalle und Pausen genauer aus den vorliegenden Daten erstellt und der Ausgabe eine Beschreibung hinzufügt.

% Mit einer Schnittstelle zur Trainingsplanerstellung Oberfläche/Ausgabe auch maximale Stunden für jeden Wochentag: h/Tag

% \subsection{Einfluss des Wettkampfdatums}
% Aktuell ist der Starttag immer auf einen Montag festgelegt. Auch wenn 
% Wie die optimale Leistungsbelastung bestimmen ohne Leistungsmesser? Ist das überhaupt nötig, wenn keine Trainingskontrolle stattfinden soll? Test um Leistungsstand zu ermitteln?
%Intensität in einer Trainingseinheit variieren? Angaben zur Intensitätsstufe/min -> ein Pool von möglichen Trainings empfehlen -> auch nicht, sondern Trainingsabschnitte nach Trainingsmethoden definieren
%\item Übertraining verhindern durch Regenerationssteuerung. Es werden keine Messungen/Anpassung im Trainingsverlauf vorgenommen, sodass Übertrainingszustände durch Einplanen von ausreichenden Regenerationsphasen verhindert werden sollen.
%Leistungsnachlass durch zu lange Pausen/Regenerationsphasen verhindern (weniger als 5 Tage und 10 Tage)
%\item \TODO{Progression an Trainingsminuten eigentlich ja nicht Belastung dann oder? Also Intensität wäre nötig!}
%\end{itemize}
%\begin{itemize}
%    \item nur x aufeinanderfolgende Tage
%    \item GA Trainings am Besten als Block \cite[34]{Radsporttraining}
%\end{itemize}

%Mithilfe dieser Informationen kann bei der Ausgabe der Trainigseinheiten näher auf die Steuerparameter (Herzfrequenz) der Leistungsbereiche eingegangen werden.  
%Anderweitige Belastungen, ungeplante Ausfälle. Wie wirkt sich Training außerhalb des Radsports aus? z.B. Ein Fußballer, der auch Rennrad fährt. \par
 
%Grund dafür ist die Diskretisierung der Leistungsbereichanteile auf fünf Minuten und die der Trainingseinheiten auf viertelstündliche Vorgaben, aber auch das Limitieren des Suchprozesses auf \TODO{Sekunden} Sekunden.
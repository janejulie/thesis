\chapter{Informatische Grundlagen}
\label{sec:grundlagen:info}
\section{Constraint Programming}
Um das Problem der Trainingsplanung zu lösen wird Constraint Programming verwendet.\cite{ConstraintProgrammierung} Dieses Programmierparadigma erweitert die logische Programmierung um Constaints. Ohne eine konkrete Lösungsstrategie algorithmisch angeben zu müssen, kann nach einer Lösung des beschriebenen Problems gesucht werden. Dafür wird das Problem in Variablen und Constraints formalisiert. Anschließend modelliert ein Solver die Lösung zum Problem.
\subsection{Variablen und Constraints}
Diese namensgebenden Constraints beschreiben die Zusammenhänge und Beziehungen in prädikatenlogischen Aussagen. Die ordentliche Beschreibung der Domain einer Variable kann den Suchraum verkleinern und die Problemlösung performanter gestalten.
\subsection{Solver}
Um eine Lösung zu finden, die alle Constraints erfüllt kommt ein Solver zum Einsatz. Das Lösen des Problems wird auch Modelling genannt. Gegebenenfalls ist keine Lösung für das Constraint-System möglich. 
\subsection{Java choco-solver}
Für diese Arbeit, die in Java \cite{arnold2005java} implementiert wird, kommt das Framework choco-solver \cite{ChocoSolverWeb} zum Einsatz. Java bietet native keine Constraint Programmierung an. Bereits bei der Arbeit zum Laufsport \ref{sec:verwandt} wurde dieses Framework verwendet, sodass es naheliegend ist ebenfalls mit Java und dem choco-solver zu arbeiten.
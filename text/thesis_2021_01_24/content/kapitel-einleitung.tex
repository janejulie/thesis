\chapter{Einleitung}
\label{sec:einleitung}
\section{Motivation}
Im Amateursport sowie im Freizeitsport steht nur in seltenen Fällen eine persönliche Trainingsbetreuung zur Verfügung. Dennoch ist auch in diesem Bereich die Effektivität des Trainings abhängig von der Trainingsplanung. Neben der Optimierung der physiologischen Leistung bringt ein Trainingsplan auch psychologische Vorteile mit sich \TODO{Quelle!}. Die Vorgabe der Trainingszeiten kann zu einer höheren Verpflichtung und gesteigerter Motivation führen, sodass die geplanten Einheiten umgesetzt werden.
Greift man auf vorgefertigte Trainingspläne zurück, büßt die Individualität des Trainingsplans ein. Die Qualität vorgefertigter Trainingspläne variiert stark je nach Quelle und ist ohne trainingswissenschaftliche Kenntnisse nicht einzuschätzen. Fehlen diese Kenntnisse, ist auch die Anfertigung eines eigenen Trainingsplans zeitaufwändig. 
So entsteht ein Bedarf nach einer Modellierung auf Grundlage der radsportspezifischen Trainingswissenschaft, die die Erstellung eines individuellen Trainingsplans für Radsportler und Radsportlerinnen übernimmt. Sie personalisiert nach Trainingsziel und Einschränkungen im Trainingsumfang. 
Bereits bestehende Systeme sind nicht speziell für den Radsport entwickelt, haben keine individuelle Anpassung der Trainingseinheiten oder sind Teil von kostenintensiven Programmen.

\section{Problemstellung}
\label{sec:einleitung:problem}
Ziel der Arbeit ist ein Programm zur automatisierten Erstellung von individuellen Trainingsplänen für den Radsport.
Das System modelliert auf Grundlage der allgemeinen und radsportspezifischen Trainingswissenschaften die Gestaltung der Trainingseinheiten einer drei- bis fünfmonatigen Aufbauphase, die periodisiert und zyklisiert ist. Die Länge des Plans ist individuell wählbar.
Typischerweise dient diese Phase zur Vorbereitung auf einen Wettkampf. Die physiologische Leistung wird vorwiegend durch die Abstimmung der Trainingseinheiten auf das Trainingsziel gesteigert. Im Straßenradsport betrifft das Trainingsziel immer den Bereich der Langzeitausdauer. Aus den unterschiedlichen Wettkampfsarten kann dennoch eine präzisere Gewichtung der Leistungsbereiche geschlussfolgert werden. Entscheidend ist das Traininigsziel also auch für ambitionierte Freizeitsportler, die anschließend keinen Wettkampf absolvieren um den Trainingsplan zu personalisieren.\newline
Eine weiterer Einfluss auf die Trainingseinheiten ist der maximal verfügbare wöchentliche Trainingsumfang. Die verfügbaren Trainingstage und -stunden sind optimal zu nutzen ohne sie zu überschreiten. 
Trotz der vielschichtigen Leistungsfaktoren behandelt diese Arbeit ausschließlich die physische Leistungssteigerung. Andere Einflussfaktoren auf die Leistung wie psychisches Belastungstraining, taktische Ausbildung, Ausrüstung oder Fahrbeschaffenheit \cite[13-15]{Radsporttraining} sind nicht Rahmen dieser Arbeit.

\section{Zielgruppe}
Diese Modellierung richtet sich in erster Linie an den Amateursport und ambitionierten Breitensport. Professionellen Sportler:innen steht im Normalfall Betreuungspersonal zur Verfügung. Bei der Fülle an Wettkämpfen in einer Saison ist in der Regel einem Wettkampf keine einzelne Aufbauphase gewidmet. Dieses System ist gerichtet an Sportler:innen, die ohne externe Betreuung trainieren. Für ambitionierte Hobbyfahrer, die im Anschluss an den Trainingsplan nicht an einem Wettkampf teilnehmen kann dieser Plan für die Planung der Radsaison dienen. Dabei entsprechen die Wettkampfarten dem Trainingsziel des Sportlers.

\section{Überblick}
\label{sec:intro:ueberblick}
\textbf{\fullref{sec:verwandt}} \\[0.2em]
Ein Überblick über bereits bestehende Systeme und deren Eigenschaften. Verglichen wird nach den Kriterien Personalisierungsgrad, Komplexität und Funktionsumfang sowie den anfallenden Kosten für den Benutzer.

\textbf{\fullref{sec:grundlagen:rad}} \\[0.2em]
Kapitel \ref{sec:grundlagen:rad} befasst sich mit den Grundlagen der Sportwissenschaften sowohl im allgemeinen Bereich als auch den radsportspezifischen Anforderungen. Trainingseinheiten werden im Kontext der Periodisierung und Zyklisierung unterschiedliche Trainingsziele erfüllen und so verschiedene Belastungsbereiche abdecken. Die verschiedenen Traninigsmethoden einer strukturierten Traninigseinheit werden vorgestellt.

\textbf{\fullref{sec:grundlagen:info}} \\[0.2em]
Die Modellierung erfolgt nach dem Programmierparadigma Constraint-Programming. Dieses Verfahren verwendet Variablen und Constraint zur Beschreibung des Systems. Eines Solver berechnet bei Existenz einer Lösung die Lösungsinstanz. 

\textbf{\fullref{sec:modellierung}} \\[0.2em]
Aus den Grundlagen der Trainingswissenschaft wird der optimale Trainingsplan modelliert. Diese Anforderungen werden im Schema der Constraint Programmierung beschrieben. Das mathematischen Modell definiert die benötigten Variablen und Constraints.

\textbf{\fullref{sec:design}} \\[0.2em]
Vor der Implementierung mit Java, wird das System konzipiert. Im UML Diagram finden sich die verwendeten Entwurfsmuster wieder. Vorrangig geht es um die Einbindung des Modells in das objektorientierte Programm. Zusätzlich entsteht die Oberfläche um die Daten zu visualisieren, die die API anbietet. \TODO{API/JSON Objekt/Server}

\textbf{\fullref{sec:implementierung}} \\[0.2em]
Die konkrete Umsetzung des Softwareprojekts wird als Java-Anwendung zur Verfügung gestellt. Das Modell wird mit der Java-Bibliothek choko-solver implementiert. \TODO{Frontend, Backend}

\textbf{\fullref{sec:zusammenfassung}} \\[0.2em]
Abschließend wird die Übertragbarkeit auf andere Sportarten aber auch die Individualität der generierbaren Trainingspläne diskutiert. \TODO{Individualität nicht vergessen im Text!}
Ausblick sowie Entstehungsprozess sind Teil dieses Kapitels. 
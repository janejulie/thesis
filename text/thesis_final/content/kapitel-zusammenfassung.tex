\chapter{Zusammenfassung}
\label{sec:zusammenfassung}
\section{Ergebnis der Arbeit}
\label{sec:zusammenfassung:ergebnis}
Die Arbeit hat gezeigt, dass die Erstellung eines Trainingsplans durch Constraint Programmierung lösbar ist. Zu diesem Zweck sind die trainingswissenschaftlichen Anforderungen definiert worden. Der Plan beachtet die Prinzipien der Zyklisierung, Periodisierung und progressiven Belastung. Er plant die Aufbauperiode in Abhängigkeit der angegebenen Wettkampfdisziplin. Welche Belastungsbereiche des Radsports abgedeckt werden, steht damit in einem direkten Zusammenhang. \par 
Zusammengesetzt wird ein Plan nach dem Baukastenprinzip. Die dynamisch definierten Trainingseinheiten sind Bausteine und haben verschiedene Anteile an den Belastungsbereichen. Das mathematische Modell optimiert die Auswahl der Einheiten, sodass die Anteile der Belastungsbereiche im Plan denen des Trainingsziels entsprechen. Begrenzt wird der wöchentliche Trainingsumfang in Stunden und Tagen. Auch die Wahl der Trainingsmethoden nimmt hierbei Einfluss, um die Variation im Trainingsplan zu gewährleisten.\par
Die Modellierung ist in eine eigenständige Anwendung umgesetzt worden. Eine übersichtliche grafische Benutzeroberfläche stellt die Funktionen zur Verfügung. Auf Ebene der Mesozyklen parallelisiert die Software das Lösen der einzelnen Monate. Zusammengesetzt ergeben diese den vollständigen Trainingsplan der Aufbauperiode. Das Programm listet die Trainingseinheiten nach der Berechnung und ermöglicht es, sie als PDF-Dokument zu exportieren. \par
Anhand beispielhafter Pläne bestätigt sich die Praktikabilität der Anwendung. Die gemessenen Abweichungen in den einzelnen Bereichen sind vertretbar, denn die Diskrepanz zum Zielwert beträgt pro Belastungsbereich unter 1\% der Trainingsminuten. Die Anwendung terminiert in angemessener Zeit und deckt die Zielgruppe aus Freizeitsportler:innen und Amateursportler:innen erfolgreich ab.

\section{Ausblick}
\label{sec:zusammenfassung:ausblick}
Obwohl die Erstellung eines Trainingsplans erfolgreich umgesetzt wurde, gibt es Möglichkeiten zur Verbesserung und Weiterentwicklung. Auf die verschiedenen Ansätze wird nachfolgend eingegangen.

\subsection{Präzisierung}
In dieser Modellierung geben die wichtigen Grundlagen der Trainingswissenschaft die Bedingungen vor. Dennoch deckt das Modell nicht alle Details ab, die Einfluss auf die Qualität eines Trainingsplans haben. Ein exaktes System ist in der Form zwar durch die Natur der Trainingswissenschaft nicht möglich, dennoch gibt es weitere etablierte Trainingsprinzipien. \par
Durch das Aufnehmen weiterer Abhängigkeiten steigt besonders die Individualität des Plans. Mögliche Größen mit Einfluss auf den Trainingsplan sind zu untersuchen. Darunter fallen der Einfluss des Alters bzw. des Trainingsalters \cite[181]{EinfuerungTrainingswissenschaft}, des Geschlechts und der Leistungsgruppe \cite[S. 173]{Radsporttraining}. Nach dem Prinzip der Superkompensation wäre es wünschenswert die Trainingseinheiten inhaltlich aufeinander abzustimmen, in denen die Blöcke einer Woche definiert werden. Diese fokussieren dann einen bestimmten Belastungsbereich. Für den Grundlagenausdauerbereich sind zum Beispiel Blöcke von drei bis fünf Tagen vorgesehen. \par
Darüber hinaus ist sogar ein vorangestellter Leistungstest vorstellbar. Mit dieser Art der Leistungsdiagnostik erfasst man die aktuelle Leistungsfähigkeit in den einzelnen Belastungsbereichen. Defizite können --  wie die Wettkampfdisziplin -- Einfluss auf die Priorisierung der Bereiche nehmen. Das hohe Maß an Individualität, das dadurch gewonnen wird, muss gegen den erhöhten Nutzeraufwand abgewogen werden. \par
In der Constraint Programmierung besteht die Möglichkeit Soft-Constraints zu definieren. Deren Erfüllung ist optional und bei einer Lösungsinstanz nicht immer gegeben. Falls die Randbedingung nicht erfüllt ist, wird die Instanz bei der Optimierung schlechter bewertet. Der Choco Solver unterstützt diesen Mechanismus nicht direkt. Jedoch erlaubt er über Reified-Constraints den Status der Erfüllbarkeit abzufragen. Addiert man die Anzahl der nicht erfüllten Bedingungen auf den Optimierungswert, rekonstruiert es die gleichen Effekte wie ein Soft-Constraint. Interessant ist dies wiederum bei den Empfehlungen zur Trainingsplangestaltung. Zum Beispiel rät man zu einem Erholungstag vor Einheiten, die einen großen Anteil an K123- oder K45-Belastungen beinhalten. Auch die Verteilung der Regenerationstage kann damit gleichmäßiger erfolgen. 

\subsection{Erweiterung} 
Durch den modularen Aufbau des Programmcodes bietet die Arbeit eine Grundlage für die Weiterentwicklung. Denkbar sind andere Ausdauersportarten und die Ausweitung der Trainingsziele. Zudem ist die Erweiterung der Makrozyklen auf die Vorbereitungsperiode und Übergangsperiode möglich. Von Vorteil ist, dass \texttt{Macro} bereits als abstrakte Klasse definiert ist. Bei der Erstellung der \texttt{Meso}-Instanzen definiert man die Zielwerte entsprechend der Periode anders. Weitere Änderungen an der Implementierung würden nur das Bereitstellen der Optionen über die Oberfläche betreffen. Auf diese Weise wären die Bausteine für einen Jahresplan vorbereitet.\par
Die Individualität des Plans kann mit der Auswahl an Wochentagen gesteigert werden. Die Benutzer:innen limitieren die Wochentage, an denen Einheiten eingeplant werden können. Die übrigen Tage müssen dann automatisch mit Pause vorbelegt sein. Das ermöglicht die bessere Abstimmung des Trainings auf die persönlichen Zeitvorgaben.
    
\subsection{Performance}
Diese Arbeit beschränkt die Erstellung des Plans auf maximal 15 Sekunden. Dieses Limit wurde zu der Modellierung hinzugefügt, um ein anwendbares System zu erhalten. Durch die zeitliche Begrenzung des Lösungsprozesses toleriert die Lösung auch eine minimale Abweichung von den Belastungsbereichen. Obwohl die Werte in einem hinreichend kleinen Rahmen sind, kann das bei der Weiterentwicklung des Programms zu Problemen führen. \newline
Eine eigens definierte Suchstrategie kann dem entgegenwirken. Mit der Flexibilität des freien Fahrtspiels kann die Distanz verringert werden. Eine geeignete Strategie würde auch die aktuelle Implementierung begünstigen, da sie die Laufzeit der Lösungssuche verkürzen kann. Sowohl eine Steigerung der Präzision als auch die Minderung der Wartezeit wären damit möglich.

\subsection{Zugänglichkeit}
Im Rahmen dieser Arbeit liegt der Fokus auf der Modellierung eines optimierten Trainingsplans, weshalb sie als eigenständige Anwendung umgesetzt ist. Um das System der Zielgruppe zur Verfügung zu stellen, ist eine Webanwendung zweckmäßiger. 
Durch die Definition einer Schnittstelle mithilfe von JSON-Objekten, kapselt sich die Modellierung von der Benutzerinteraktion ab. Nötig ist hierfür das Deployment der Anwendung auf einem Server. Im Anhang \ref{anhang:json} wird eine Möglichkeit für eine Schnittstelle über ein JSON-Objekt dargestellt.

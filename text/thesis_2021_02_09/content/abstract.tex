\pdfbookmark[0]{Zusammenfassung}{Abstract}
\chapter*{Zusammenfassung}
\label{sec:abstract}
\vspace*{-10mm}
Besonders im Amateurbereich spielt die Vorbereitungsphase zu den Wettkämpfen eine wichtige Rolle. Die maximale Leistung auf dem Rad wird vorallem von der physischen Leistungsfähigkeit beeinflusst. Unterschiedliche Trainingsziele nehmen Einfluss auf die Trainingsinhalte eines individuellen Sportlers. Mit personalisierten Trainingsplänen soll ein optimaler Trainingsverlauf für eine breite Personengruppe ermöglicht werden.
Um die Leistung zu steigern, wird diese Modellierung nach Erkenntnissen der allgemeinen und radsportspezifischen Trainingswissenschaft die Trainingseinheiten aufeinander abstimmen und so strukturieren, dass alle relevanten Leistungsbereiche abgedeckt werden.
\vspace*{20mm}

{\usekomafont{chapter}Abstract}\label{sec:abstract-diff} \\
\vspace*{-10mm}

\chapter{Implementierung}
\label{sec:implementierung} 
Einbettung des Solvers in Java Programm
\begin{itemize}
    \item Inhalte der JSON-Objekte (siehe \ref{sec:modellierung:output})
    \item API verfügbar für andere Anwendungen?
    \item Parallelisierung der Mesozyklen
\end{itemize}

\section{Optimierung}
\subsection{Constraint Programmierung in JAVA}
Die Programmiersprache JAVA\cite{java} bietet nativ keine Constraint Programmierung an. Mit Choco \cite{ChocoSolverWeb}, einer Open-Source Bibliothek ist die Modellierung von Lehrprojekten und eigene Projekten möglich. Alle oben genannten Funktionen werden unterstützt.
\subsection{Modellierung mit choko-solver}
Bei der Lösung des Optimierungsproblems kommt das Framework choco-solver zum Einsatz. Die Optimierung wird ausgelöst auf jeden Monat. Um Laufzeit zu sparen ist die Optimierung der einzelnen Monate parallelisiert.

\section{Frontend}

\begin{itemize}
    \item Darstellung der Trainingseinheiten als Liste in der Java Applikation
    %\item Implementierung verfügbar unter \url{http://www.tba.com}
\end{itemize}

\section{Übertragbarkeit}
Die Grundlage dieser Arbeit war eine vorangegangene Bachlorarbeit, die der Laufsport betrifft. Mit Ausblick auf einer weiteren Arbeit zur Modellierung des Schwimmtrainings. Vorstellbar ist die Kombination dieser Arbeiten zu einem Trainingsplan für Triathleten. 
Aus diesem Grund ist die Arbeit modular gegliedert.
Für viele Sportarten gelten die Trainingsprinzipien der Zyklisierung, Periodisierung, progressive Belastung und Superkompensation. Diese Struktur kann für andere Ausdauersportarten übernommen werden. Um die Trainingsplanerstellung für andere Sportarten zu modellieren, sind folgende Erweiterungen im Code nötig.
Die Definition der Leistungsbereiche ist eine Implementierung des Interfaces. Genauso ist die Liste der Trainingsmethoden der Sportart eine andere. Auch diese ist als Interface definiert. 
\begin{itemize}
    \item Übertragbarkeit auf andere Sportarten? Welche Constraints sind speziell für den Radsport vs. allgemeine Trainingsprinzipien
    \item modulare Entwicklung der Anwendung?
\end{itemize}
\chapter{Zusammenfassung}
\label{sec:zusammenfassung}
\section{Ergebnis der Arbeit}
\label{sec:zusammenfassung:ergebnis}
Chronologische Zusammenfassung vorheriger Kapitel

\section{Übertragbarkeit}
\begin{itemize}
    \item Übertragbarkeit auf andere Sportarten? Welche Constraints sind speziell für den Radsport vs. allgemeine Trainingsprinzipien
    \item modulare Entwicklung der Anwendung?
\end{itemize}

\section{Ausblick}
\label{sec:zusammenfassung:ausblick}
\begin{itemize}
    \item Wie wirkt sich Training außerhalb des Radsports aus? z.B. Ein Fussballer, der auch Rennrad fährt.
    \item Zusammenschluss der Arbeiten dann zu einem Trainingsplan spezifisch für die Triathlon Vorbereitung möglich?
    \item Erweiterung zur Trainingskontrolle und Trainingsdokumentation
    \item Genauere Planung der Einheiten
    \begin{itemize}
        \item Trainingsjahr statt Alter -> out of scope \cite[181]{EinfuerungTrainingswissenschaft}
        \item Geschlecht: Welche Auswirkungen hat das?
        \item Leistungsgruppe \cite[S. 173]{Radsporttraining} (unregelmäßig Trainierende, Ausgleichssportler, Hobbyfahrer ohne Wettkampf, Fahrer mit Wettkämpfen) -> Unterscheiden sich hauptsächlich in "maximaler wöchentlicher Umfang"
        \item Gesamtdauer des Plans
            \begin{itemize}
                \item Jahresplan
                \item einzelner Makrozyklus
            \end{itemize}
    \end{itemize}   
    \item Parallelisierung der Mesozyklen
\end{itemize}

\section{Notizen}
\begin{itemize}
    \item Wie die optimale Leistungsbelastung bestimmen ohne Leistungsmesser? Ist das überhaupt nötig, wenn keine Trainingskontrolle stattfinden soll? Test um Leistungsstand zu ermitteln?
    \item Intensität in einer Trainingseinheit variieren? Angaben zur Intensitätsstufe/min -> ein Pool von möglichen Trainings empfehlen -> auch nicht, sondern Trainingsabschnitte nach Trainingsmethoden definieren
    \item Übertraining verhindern durch Regenerationssteuerung. Es werden keine Messungen/Anpassung im Trainingsverlauf vorgenommen, sodass Übertrainingszustände durch Einplanen von ausreichenden Regenerationsphasen verhindert werden sollen.
    \item Leistungsnachlass durch zu lange Pausen/Regenerationsphasen verhindern (weniger als 5 Tage und 10 Tage)
\end{itemize}
